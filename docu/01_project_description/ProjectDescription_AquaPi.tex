\documentclass{../../git_submodules/common_docu/doc_class}
\newcommand{\ThisProjectName}{AquaPi}
\newcommand{\defaultAuthor}{Cypax}

\project{\ThisProjectName}
\author{Cypax}
  
\author{\defaultAuthor}
\date{2019-04-04}
\project{\ThisProjectName}
%\projectsub{subproject} % i.e. "GUI"
\doctitle{Project description}
\projectversion{1.0}
 
\begin{document}
  
  \section{Introduction}
  \ThisProjectName~is a control unit for an aquarium, based on a \hyperref{https://en.wikipedia.org/wiki/Raspberry_Pi}{}{}{Raspberry Pi} single board computer.
  
  It provides classical aquarium features as like timer switches for light or displaying water temperature and further customizable functions.
  
  The GUI is provided as a website, allowing to operate it remotely (e.g. via a smart phone) as well as via a connected (touch) display.
  
  \section{Intended use}
  \subsection{Functionality}
  The intended use of \ThisProjectName~is to 
  \begin{itemize}
    \item control typical aquarium  devices (light, pumps, heater, water and gas valves)
    \item display and supervise water temperature
    \item provide a user interface (remotely or locally accessed via a website)
    \item provide "service functions" -- temporarily activate/deactivate aquarium devices for special occurrences like aquarium maintenance or fish feeding.
  \end{itemize}

  \subsection{Operation}
  \ThisProjectName~is intended to run autonomously 24/7.\\
  It is to be expected that the aquarium may not be supervised by any person up to two weeks (e.g. in case of holiday absence).

  \subsection{System lifetime}
  The system lifetime is based on the durability of typical aquariums and expected to be up to 10 years.
  
  \subsection{Device quantity}
  It is expected that \ThisProjectName~will be installed in 10\textsuperscript{0} to 10\textsuperscript{1} aquariums.
  
  \section{Variants}[all]\label{variant:all}  
  There are not project variants.
  
\end{document}
