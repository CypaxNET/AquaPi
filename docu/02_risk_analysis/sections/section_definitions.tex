\section{Risk definitions}

\subsection{Severity}
Concerning risk severity it has to taken into account that \ThisProjectName~ controls an aquarium and risks therefore may apply to 
\begin{itemize}
  \item persons -- user(s) and third party persons,
  \item animals -- inside the aquarium,
  \item material values -- e.g. the aquarium itself, aquarium devices, environment, \dots.
\end{itemize}

\begin{table}[H]
  \begin{tabularx}{\textwidth}{|l|X|}
    \hline
    \textbf{Severity} & \textbf{Definition} \\
    \hline
    \textbf{insignificant} & Persons: Inconvenience or temporary discomfort. \newline
    Animals: Self-healing injury or impairment. \newline
    Material values: Loss of value \textless{} 1 \euro. \\
    \hline
    \textbf{minor} & Persons: Injury or impairment not requiring medical intervention. \newline
    Animals: Irreversible injury or impairment.\newline
    Material values: Loss of value \textgreater= 1 \euro{} .. \textless{} 10 \euro. \\
    \hline
    \textbf{moderate} & Persons: Injury or impairment requiring medical intervention. \newline
    Animals: Death of one animal. \newline
    Material values: Loss of value \textgreater= 10 \euro{} .. \textless{} 100 \euro. \\
    \hline
    \textbf{serious} & Persons: Injury or impairment requiring professional medical intervention. \newline
    Animals: Death of several animals. \newline
    Material values: Loss of value \textgreater= 100 \euro. \\
    \hline
    \textbf{critical} & Persons: Permanent impairment or life-threatening injury. \\
    \hline
    \textbf{catastrophic} & Persons: Death.\\
    \hline
  \end{tabularx}
  \caption{Definition of risk severities}
\end{table}


\subsection{Probability}

\subsubsection{Probability that a hazardous situation occurs}

\ThisProjectName~ controls various aquarium devices by activating/deactivating relay switches following user configurable schedules.\\
It is reasonable to assume that any aquarium device is activated once and deactivated once per day.

Since 8 relays can be controlled, plus possibly a few further devices controlled using PWM or GPIO outputs of the Raspberry Pi, it is assumed that \NumRiskRelatedActionsPerDay~\textit{risk related actions} per day will be performed.\\
In \NumYearsLifeTime~ years of expected device lifetime that makes a total of \FPeval{\result}{clip(\NumYearsLifeTime * 365 * \NumRiskRelatedActionsPerDay)}\numprint{\result}~actions.

The graduation of risk probabilities is chosen such way, that a \textit{inconceivable} risk clearly will not occur during the expected device time while a \textit{frequently} risk will occur once a day.

\begin{table}[H]
  \begin{tabularx}{\textwidth}{|l|X|X|X|}
    \hline
	\textbf{Probability} & \textbf{Probability per action} & \textbf{Occurrences per lifetime and device} &
	\textbf{Statistically mean time between occurrence (days)} \\
    \hline
	\PrintInnerProbabilityRow{inconceivable}{$\leq$}
	\PrintInnerProbabilityRow{exceptionally}{$\leq$}
	\PrintInnerProbabilityRow{rarely}{$\leq$}
	\PrintInnerProbabilityRow{occasionally}{$\leq$}
	\PrintInnerProbabilityRow{probably}{$\leq$}
	\PrintInnerProbabilityRow{frequently}{$\leq$}
	\PrintInnerProbabilityRow{regularly}{$\leq$}
  \end{tabularx}
  \caption{Definition of risk probabilities (inner sequence of events)}
\end{table}

Parameters used for calculation:
\begin{itemize}
	\item \NumRiskRelatedActionsPerDay~risk related actions per day
	\item \NumYearsLifeTime~years device lifetime
\end{itemize}

\subsubsection{Probability that a hazardous situation results in a harm}
\begin{table}[H]
	\begin{tabularx}{\textwidth}{|l|X|}
		\hline
		\textbf{Probability} & \textbf{Probability of harm} \\
		\hline
		\PrintOuterProbabilityRow{exceptionally}{$\leq$}
		\PrintOuterProbabilityRow{rarely}{$\leq$}
		\PrintOuterProbabilityRow{occasionally}{$\leq$}
		\PrintOuterProbabilityRow{sometimes}{$\leq$}
		\PrintOuterProbabilityRow{most likely}{$\leq$}
		\PrintOuterProbabilityRow{always}{$\leq$}
	\end{tabularx}
	\caption{Definition of risk probabilities (outer sequence of events)}
\end{table}
